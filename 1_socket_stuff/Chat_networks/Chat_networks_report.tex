\documentclass[landscape]{article}

\usepackage{minted}

\usepackage{geometry}
 \geometry{
 a4paper,
 %total = {170mm,257mm},
 left = 15mm,
 %top = 20mm,
 }

\title{HTTP Server Report}
\author{Pablo Collado Soto}
\date{}

\begin{document}

    \maketitle

    \section{Client - Server Architecture}
        \subsection{Server}
            \begin{minted}[linenos]{python}
######################################################
#                   Imports                          #
# socket -> Let's us manage socket instances         #
# sys -> Grant access to the passed arguments        #
# select -> Let's us work concurrently with sockets  #
# os -> Let's us clear the screen when printing info #
######################################################

# NOTE: As these imports are always the same we'll just talk about them here so as not to bloat the document!

import socket, sys, select, os

##################################################################################################################
#                                  Closing sockets and the TIME_WAIT interval                                    #
# As seen in https://serverfault.com/questions/329845/how-to-forcibly-close-a-socket-in-time-wait and in         #
# RFC 793 (https://tools.ietf.org/html/rfc793), when closing a TCP socket it will remain "open" long             #
# enough to guarantee the FIN ACK segment has reached the opposite end. If we close the clients then             #
# the random port they've reserved will be held for some time. As clients are not binded to a specific port      #
# this will have no impact whatsoever when launching new clients, they'll just get a new port number and life    #
# will seamlessly go on. If we manually close the server however the port we had binded it to will continue      #
# to be used for TIME_WAIT seconds... We can check that by running netstat and looking at the top of the         #
# output. Lazy people like me would instead run netstat | head -n 20 and get away with it. We can configure      #
# this TIME_WAIT to be 0. I'm not sure whether that's done on form python on the socket or from the OS on the    #
# network manager... Anyway we should not forget it's there and that, even if we shut down our sockets correctly #
# we will have an inherent delay due to how TCP works... When we close every client gracefully and end up        #
# shutting the server down we won't experience this behaviour as the server is only left with the "listening"    #
# socket, there are NO connections open so there is no need to wait for them to receive the final message!       #
##################################################################################################################

##################################################################################################################
#                                          Allowing Outside Connections                                          #
# In order to allow connections from the outside we must bind the server to every interface or, at least the one #
# providing connectivity to the outside world. If we only bind to that "outwards" interface we won't be able to  #
# connect local clients... We can then choose to bind to '' instead of '127.0.0.1' which is interpreted as       #
# INADDR_ANY by Pyhton, that is, we'll bind the server to EVERY interface                                        #
##################################################################################################################

def main():
    if len(sys.argv) != 2:
        print("Use: {} <binding_port>".format(sys.argv[0]))
        exit(-1)

    # The readable_sources dictionary contains the different socket instances as values and some information identifying
    # them as the key. These keys will be derived from the first message sent by each client as soon as they connect.
    readable_sources = {'serving_sock': socket.socket(socket.AF_INET, socket.SOCK_STREAM)}

    # Bind the socket and get ready for incoming connections
    readable_sources['serving_sock'].bind(('127.0.0.1', int(sys.argv[1])))

    print("Listening for connections...")

    readable_sources['serving_sock'].listen(1)

    # These variables let us monitor images being sent through us so that we can
    # act accordingly
    img_passing = False
    passed_bytes = 0

    try:
        while True:
            for src in select.select(list(readable_sources.values()), [], [])[0]:

                # If we got a new client accept it and read its ID Tag. Note reading on a socket
                # not monitored by select() is not such a good idea as it may block our server
                # indefinitely but as we have implemented both the client and server we can
                # get away with this little "botch"
                if src == readable_sources['serving_sock']:
                    new_client = src.accept()[0]
                    readable_sources[new_client.recv(8148).decode()] = new_client

                # If the message came from a socket connected from a client
                else:
                    inc_msg = src.recv(8148)

                    # If there is no image traversing us at the moment we can safely decode the message.
                    # Otherwise we could be attempting to decode some invalid bytes and that would trigger
                    # an exception...
                    if not img_passing:

                        # If the peer socket's been closed close this end and remove it from the monitored list
                        if inc_msg.decode() == '':
                            src.close()

                            # As we are working with dictionaries so that we can print more meaningful information we need
                            # to jump thorugh some hoops to get the correct value to delete...
                            del readable_sources[list(readable_sources.keys())[list(readable_sources.values()).index(src)]]

                        # Our image sending protocol starts with an 'IMG_INCOMING!' message. It that's the case activate
                        # the "image mode" until the given number of bytes have passed
                        elif 'IMG_INCOMING!' in inc_msg.decode():
                            img_passing = True
                            passed_bytes = int(inc_msg.decode().split(';')[1])

                    # We are currently transmitting an image
                    else:

                        # Keep track of all the sent bytes
                        passed_bytes -= len(inc_msg)

                        # If we are done deactivate "image mode"
                        if passed_bytes == 0:
                            img_passing = False
                    
                    # As the data is of no interest to us we don't need to decode() it. We can just send it away
                    for c_sock in list(readable_sources.values())[1:]:
                            if c_sock != src:
                                c_sock.sendall(inc_msg)

            # As images can be large printing information on the screen at such a high rate makes the terminal flicker
            # so avoid printing any more information whilst an image is being sent
            if not img_passing:

                # Take advantage of the informative tags sent by the clients so that we can show meaningful info at the output!
                os.system("clear")
                print("Connected clients:")
                for client, c_sock in list(readable_sources.items())[1:]:
                    print("\t{} -> {}:{}".format(client, c_sock.getpeername()[0], c_sock.getpeername()[1]))

    # When we receive CTRL + C
    except KeyboardInterrupt:
        print("Quitting...")

        # Close every socket and quit!
        for sock in readable_sources.values():
            sock.close()
        exit(0)

if __name__ == '__main__':
    main()
            \end{minted}

        \subsection{Client}
            \begin{minted}[linenos]{python}
import socket, sys, select, time

#####################################################################
#                           Imports                                 #
# time -> Let's us introduce the needed delays when sending images! #
#####################################################################

# NOTE: The socket module has a sendfile() method but we decided to manually do it ourselves.
# NOTE: At least for the first time...

#############################################################################################
#                              How cool are ANSI Escape Sequences?                          #
# ANSI Escape Sequences are, as their name implies, special sequences that we can print     #
# to a terminal to control its behavior instead of showing characters on screen. You can    #
# find a very comprehensive table over at https://en.wikipedia.org/wiki/ANSI_escape_code.   #
# We have used the following sequences which can be found in the aforementioned site:       #
#           ESC[nF -> Move the cursor to the beginning of the line n lines up               #
#           ESC[nG -> Move the cursor to column n in the current line                       #
#           ESC[2K -> Erase the entire line without changing the cursor¡s position          #
# Note the ESC character is given by 0x1B or 27 as seen in the ASCII table, that's why      #
# you see the leading \x1B in these sequences. Tradition has it that we use the octal       #
# equivalent, \033, but we are more of the "hexy" kind... Combining these escape sequences  #
# with the program flow we have amanged to build a more visually appealing CLI for our chat #
# client but it's still not as robust as we would like it to be...                          #
#############################################################################################

def main():
    if len(sys.argv) != 4:
        print("Use: {} <server_IP> <server_port> <username>".format(sys.argv[0]))
        exit(-1)
    elif sys.argv[3] == 'serving_sock':
        print("This username is restricted... Use another one!")
        exit(-1)

    # CLI Stuff
    prompt = 'Type something --> '
    clear_prompt = "\x1B[1G" + "\x1B[2K"
    clear_input = "\x1B[1F" + "\x1B[2K"

    # We could also use this clear_prompt sequence depending on our taste!
    # clear_prompt = "\x1B[" + str(len(prompt)) + "D" + "\x1B[2K"

    # Set up our connection socket
    info_sources = [socket.socket(socket.AF_INET, socket.SOCK_STREAM), sys.stdin]
    info_sources[0].connect((sys.argv[1], int(sys.argv[2])))

    # Send the username as soon as you connect so that the server can move on!
    info_sources[0].sendall(sys.argv[3].encode())

    print("We're connected to -> {}:{}".format(sys.argv[1], sys.argv[2]))

    # Variables for controlling image sending and reception
    inc_file = None
    reading_file = False
    rcv_bytes = 0

    try:
        while True:

            # If we are receiving an image don't write the prompt. Note we are flushing it because there
            # is no trailing newline ('\n'). Most terminals won't clear the buffers until they see one...
            if not reading_file:
                print(prompt, end = '', flush = True)
            for src in select.select(info_sources, [], [])[0]:

                # If the user has typed a message
                if src == sys.stdin:

                    # Read it
                    inc_msg = sys.stdin.readline().rstrip()

                    # If they want to send an image
                    if "send_img" in inc_msg:

                        # Open it up for reading IN BINARY after getting the filename
                        f_name = inc_msg.split(' ')[1]
                        img = open(f_name, 'rb')
                        img_data = img.read()

                        # Inform the server and clients that you're about to send an image and its lenght
                        info_sources[0].sendall(("IMG_INCOMING!;" + str(len(img_data))).encode())

                        # Wait to make sure TCP buffers are cleared!
                        time.sleep(0.1)

                        # Send all the data
                        info_sources[0].sendall(img_data)

                        # Close the file and move on
                        img.close()

                    # If we are not sending an image
                    else:
                        # Build the output message by appending your username
                        out_msg = sys.argv[3] + " -> " + inc_msg

                        # Print the message to your own screen
                        print(clear_input + out_msg)

                        # Ans send it to the server
                        info_sources[0].sendall(out_msg.encode())

                # If we are receiving a message
                else:
                    # And it's not a file
                    if not reading_file:

                        # We can safely decode it
                        i_msg = info_sources[0].recv(8148).decode()

                        # The server has either kicked us out or gone down, just quit
                        if i_msg == '':
                            raise KeyboardInterrupt

                        # If an image is about to come
                        elif 'IMG_INCOMING!' in i_msg:

                            # Get the total incoming bytes
                            rcv_bytes = int(i_msg.split(';')[1])

                            # Tell the user
                            print(clear_prompt + "Image incoming!\tSize: {}".format(rcv_bytes))

                            # Open a file for storing the incoming data
                            inc_file = open("rcv_img.png", 'wb')

                            # And activate the reading mode, again, to prevent decoding illegal bytes
                            reading_file = True

                        # If there is no file coming just print the message
                        else:
                            print(clear_prompt + i_msg)

                    # While we are reading an image
                    else:

                        # Read the data without attempting to decode it
                        i_msg = info_sources[0].recv(8148)

                        # Keep track of the received bytes
                        rcv_bytes -= len(i_msg)

                        # Write the data to the opened file
                        inc_file.write(i_msg)

                        # When finished just close the file and deactivate the "image mode"
                        if rcv_bytes == 0:
                            reading_file = False
                            inc_file.close()

    # If we receive CTRL + C or the server tears down our connection
    except KeyboardInterrupt:
        print("Quitting...")

        # Close our socket and just quit
        info_sources[0].close()
        exit(0)

if __name__ == "__main__":
    main()
            \end{minted}

    \section{Peer-to-peer \texttt{P2P} Architecture}
        \subsection{Welcome Server (\texttt{UDP} based)}
            \begin{minted}[linenos]{python}
import socket, select, sys, os

##################################################################
#                   SIGNALLING MESSAGE FORMATS                   #
# Welcome Message Format -> "I'm up!;username;entry_socket_port" #
# Peer Address Format    -> "IP:Port"                            #
# Goodbye Message Format -> "Bye!;username"                      #
##################################################################

def main():
    if len(sys.argv) != 2:
        print("Use: {} <binding_port>".format(sys.argv[0]))
        exit(-1)

    # We're using UDP for sending and receiving messages
    welcome_soket = socket.socket(socket.AF_INET, socket.SOCK_DGRAM)

    # Begin listening for new and departing peers
    welcome_soket.bind(('127.0.0.1', int(sys.argv[1])))

    # This dictionary contains the usernames as keys and a string of
    # the form 'IP:PORT' as values. These are the IP and PORT of
    # a peer's welcome socket, i.e, the one it accepts connections on
    active_peers = {}

    print("Waiting for new peers...")

    try:
        while True:
            # Using select here is just overkill but we want to be consistent...
            for src in select.select([welcome_soket], [], [])[0]:

                # This is the only possible source anyway... Checking it improves readability though!
                if src == welcome_soket:

                    # We'll need the sender's IP. Note ther is no use for the port number as it
                    # belongs to the UDP socket and we are only concerned with the TCP socket
                    # accepting connections. That's the one the peer will let us know about!
                    msg, source = src.recvfrom(8148)

                    # If the peer is introducing itslef
                    if "I'm up!" in msg.decode():

                        # Send info pertaining already connected peers back
                        for peer, cnx_point in active_peers.items():
                            src.sendto((peer + ';' + cnx_point).encode(), source)

                        # Then add an entry for this new peer in our dictionary
                        active_peers[msg.decode().split(';')[1]] = source[0] + ';' + msg.decode().split(';')[2]

                    # If the peer is disconnecting
                    elif "Bye!" in msg.decode():

                        # Just delete the associated entry in our dictionary
                        del active_peers[msg.decode().split(';')[1]]

            # Clear the terminal's screen
            os.system("clear")
            print("Live peers:")

            # Take advantage of the meaningful keys we are using to show information on the current network status
            for peer, cnx_point in active_peers.items():
                print("\t{} -> {}:{}".format(peer, cnx_point.split(';')[0], cnx_point.split(';')[1]))

    # If we receive CTRL + C
    except KeyboardInterrupt:
        print("Quitting...")

        # Close the socket and quit but don't tell the peers. The P2P network must function independently of this server!
        welcome_soket.close()
        exit(0)

if __name__ == "__main__":
    main()
            \end{minted}

        \subsection{Peer}
            \begin{minted}[linenos]{python}
import socket, sys, select

################################################################
#               SIGNALLING MESSAGE FORMATS                     #
# peer_info (see line 95) -> "peer_username;peer_IP;peer_port" #
################################################################

def main():
    if len(sys.argv) != 4:
        print("Use: {} <server_IP> <server_port> <username>".format(sys.argv[0]))
        exit(-1)
    elif sys.argv[3] == 'introduction_sock' or sys.argv[3] == 'entry_sock' or sys.argv[3] == 'keybrd':
        print("Invalid username!")
        exit(-1)

    # CLI Stuff
    prompt = 'Type something --> '
    clear_prompt = "\x1B[1G" + "\x1B[2K"
    clear_input = "\x1B[1F" + "\x1B[2K"

    # We'll use this list to check whether a key is associated with a peer or it is related to a "special" socket.
    # This'll let us condense code when printing peer_sockets' keys as we'll see later on as well as decide who
    # to send messages to!
    special_sockets = ['keybrd', 'introduction_sock', 'entry_sock']

    # This dictionary will hold every monitored socket plus the keyboard:
        # 'keyboard' -> The associated value is the keyboard's file descriptor
        # 'introduction_sock' -> UDP socket used to introduce ourselves to the net and get info regarding active peers
                               # We'll also use it when disconnecting to inform the server so that it forgets about us
        # 'entry_sock' -> TCP socket used for accepting new peers

    # New entries will use the peer's username as the dictionaries key and the value will be the corresponding socket instance
    peer_sockets = {'keybrd': sys.stdin,
                    'introduction_sock': socket.socket(socket.AF_INET, socket.SOCK_DGRAM),
                    'entry_sock': socket.socket(socket.AF_INET, socket.SOCK_STREAM)
    }

    # Binding to port number 0 will use any available port. The kernel will be smart enough to assign one.
    # Begin listening for new peers afterwards
    peer_sockets['entry_sock'].bind(('127.0.0.1', 0))
    peer_sockets['entry_sock'].listen(1)

    # The getsockname() method lets us get the IP and PORT of a given socket. Remember we are not using a user supplied port...
    print("We are listening for peers @ {}:{}".format(peer_sockets['entry_sock'].getsockname()[0], peer_sockets['entry_sock'].getsockname()[1]))

    # Tell the server we are up so that it takes note and sends back info about already existing peers
    peer_sockets['introduction_sock'].sendto(("I'm up!;" + sys.argv[3] + ';' + str(peer_sockets['entry_sock'].getsockname()[1])).encode(),
                                             (sys.argv[1], int(sys.argv[2])))

    print("We have introduced ourselves to the central server!\nTime to connect to the peers!")

    try:
        while True:

            # Write the prompt and flush it, remember we don't have a trailing '\n'
            print(prompt, end = '', flush = True)
            ready_socks = select.select(list(peer_sockets.values()), [], [])[0]
            for src in ready_socks:

                # If the user typed a message
                if src == sys.stdin:

                    # Read it
                    in_data = sys.stdin.readline().rstrip()

                    # If the user want's to know who he/she is connected to...
                    if in_data == 'get_status':
                        print(clear_input + "Connected peers:")

                        # Print the associated peer_sockets' keys, i.e, the usernames
                        for peer in peer_sockets:

                            # As dicitonaries don't have to be ordered we'll check the name we print doesn't belong to either the
                            # keyboard, the introduction UDP socket or the binded TCP one. This is why we defined the special_sockets
                            # list at the beginning.
                            if peer not in special_sockets:
                                print("\t" + peer)

                    # If it's a regular message
                    else:

                        # Build it by appending your own username
                        out_msg = sys.argv[3] + " -> " + in_data
                        print(clear_input + out_msg)

                        # Send the message only to peers!
                        for peer, sock in peer_sockets.items():
                            if peer not in special_sockets:
                                sock.sendall(out_msg.encode())

                # If there is a message coming to the UDP socket it must be info about a peer
                elif src == peer_sockets['introduction_sock']:

                    # Parse the incoming message
                    peer_info = src.recv(8148).decode().split(';')

                    # Add an entry in the peer_sockets dictionary
                    peer_sockets[peer_info[0]] = socket.socket(socket.AF_INET, socket.SOCK_STREAM)

                    # Tell the user we have received info about a new client
                    print(clear_prompt + "Connecting to peer {} @ {}:{}".format(peer_info[0], peer_info[1], peer_info[2]))

                    # Connect to said client
                    peer_sockets[peer_info[0]].connect((peer_info[1], int(peer_info[2])))

                    # Send our username to the new peer right away. He/she will be waiting for it!
                    peer_sockets[peer_info[0]].sendall(sys.argv[3].encode())

                # If a new peer is connecting to us
                elif src == peer_sockets['entry_sock']:

                    # Accept it
                    new_sock, new_addr = src.accept()

                    # Get it's username. Reading here is not as dangerous as it could be because all peers are implemented in the
                    # same way!
                    peer_username = new_sock.recv(8148).decode()

                    # Tell the user a new peer is connecting to us
                    print(clear_prompt + "Got new peer {} @ {}:{}".format(peer_username, new_addr[0], new_addr[1]))

                    # Add the new peer to our dictionary
                    peer_sockets[peer_username] = new_sock

                # If the message came to an already connected socket
                else:

                    # Decode it
                    inc_msg = src.recv(8148).decode()

                    # If we get an empty string (i.e EOF) it means the other end has closed the socket
                    if inc_msg == '':

                        # Get the socket's index
                        socket_index = list(peer_sockets.keys())[list(peer_sockets.values()).index(src)]

                        # Tell the user a peer has disconnected
                        print(clear_prompt + "Disconecting from peer {} @ {}:{}".format(socket_index, src.getpeername()[0], src.getpeername()[1]))

                        # Close it
                        src.close()

                        # And delete it
                        del peer_sockets[socket_index]

                    # Otherwise just print the message
                    else:
                        print(clear_prompt + inc_msg)

    # If we get a CTRL + C signal
    except KeyboardInterrupt:
        print("Quitting...")

        # Tell the server we are leaving
        peer_sockets['introduction_sock'].sendto(("Bye!;" + sys.argv[3]).encode(), (sys.argv[1], int(sys.argv[2])))

        # Close all the open sockets, including the UDP and binded TCP ones
        for peer in list(peer_sockets.values())[1:]:
            peer.close()

        # And quit
        exit(0)

if __name__ == "__main__":
    main()
            \end{minted}
\end{document}